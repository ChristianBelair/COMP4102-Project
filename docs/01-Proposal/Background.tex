\section{Background}
\subsection{Gaze Detection and Tracking}

Gaze detection and tracking has many positive implications for driver safety, namely to detect if a driver is distracted or drowsy. The US department of transportation found that in years 2011-2015 an overall $2.5\%$ of fatalities were caused by drowsy driving~\cite{CrashDrowsy}, and for distracted driving in 2018 it was found that $8\%$ of fatalities were distraction-affected~\cite{CrashDistracted}. Currently there are a wide variety of well-established and novel techniques for gaze tracking as found by a survey by Chennamma and Xiaohui~\cite{chennamma2013survey}. 

\subsection{Pedestrian Detection and Tracking}

Pedestrian detection and tracking is an important aspect in driver safety. Drivers must be aware when one or more pedestrians are moving around their vehicle as to ensure their safety. Stimpson, Wilson, and Muelleman research into pedestrian fatalities shows that from 2005 to 2010 the fatality rate of 116.1 per 10 billion vehicle miles driven had increased to 168.6 in 2010~\cite{PedestrianFatalities}.
OpenCV includes an implementation of the Histograms of Oriented Gradients (HOG) which can identify a person within an image or video. This creates a basis for identifying pedestrians.
The implementation details in terms of human detection are found in this paper\cite{HOGHumanDetection}

\subsection{Road Sign Tracking}
% https://arxiv.org/abs/2010.06453
% https://pubmed.ncbi.nlm.nih.gov/14733981/
% https://www.motorists.org/blog/nma-speed-trap-spotlight-virginia/
Road signs are vital to road safety. 
Stop sign violations accounted for approximately 70\% of all crashes.\cite{Pubmed}
In addition, in some areas, speed limits are lowered in unexpected areas pose great risk for a driver, even when they're striving to follow the speed limit.
While much research into this area is focused on using machine learning, some computer vision exclusive implementation have been studied.\cite{Ayaou2020}